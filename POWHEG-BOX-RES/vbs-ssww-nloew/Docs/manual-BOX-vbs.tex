\documentclass[paper]{JHEP3}
\usepackage{amsmath,amssymb,enumerate,url}
\bibliographystyle{JHEP}
\usepackage{lineno}
%\linenumbers

%%%%%%%%%% Start TeXmacs macros
\newcommand{\tmtextit}[1]{{\itshape{#1}}}
\newcommand{\tmtexttt}[1]{{\ttfamily{#1}}}
\newenvironment{enumeratenumeric}{\begin{enumerate}[1.] }{\end{enumerate}}
\newcommand\sss{\mathchoice%
{\displaystyle}%
{\scriptstyle}%
{\scriptscriptstyle}%
{\scriptscriptstyle}%
}

\newcommand\as{\alpha_{\sss\rm S}}
\newcommand\POWHEG{{\tt POWHEG}}
\newcommand\POWHEGBOX{{\tt POWHEG BOX}}
\newcommand\HERWIG{{\tt HERWIG}}
\newcommand\PYTHIASIX{{\tt PYTHIA6}}
\newcommand\PYTHIAEIGHT{{\tt PYTHIA8}}
\newcommand\PYTHIA{{\tt PYTHIA}}
\newcommand\PHOTOS{{\tt PHOTOS}}
\newcommand\PHOTOSPP{{\tt PHOTOS++}}
\newcommand\MCatNLO{{\tt MC@NLO}}
\newcommand\DIR{{\tt Z\_ew-BMNNPV}}

\newcommand\kt{k_{\sss\rm T}}

\newcommand\pt{p_{\sss\rm T}}
\newcommand\LambdaQCD{\Lambda_{\scriptscriptstyle QCD}}
%%%%%%%%%% End TeXmacs macros


\title{Same-sign W-boson scattering at EW NLO accuracy in the POWHEG-BOX-RES: user manual} \vfill
\author{Mauro Chiesa\\
  Universit\"at W\"urzburg, Institut f\"ur Theoretische Physik und Astrophysik\\
  Emil-Hilb-Weg 22, 97074 W\"urzburg, Germany\\
  E-mail: \email{Mauro.Chiesa@physik.uni-wuerzburg.de}}
\author{Ansgar Denner\\
  Universit\"at W\"urzburg, Institut f\"ur Theoretische Physik und Astrophysik\\
  Emil-Hilb-Weg 22, 97074 W\"urzburg, Germany\\
  E-mail: \email{Ansgar.Denner@physik.uni-wuerzburg.de}}
\author{Jean-Nicolas Lang\\
  Universit\"at Z\"urich, Physik-Institut,\\
  CH-8057 Z\"urich, Switzerland\\
  E-mail: \email{jlang@physik.uzh.ch}}
\author{Mathieu Pellen\\
  University of Cambridge, Cavendish Laboratory,\\
  Cambridge CB3 0HE, United Kingdom\\
  E-mail: \email{mpellen@hep.phy.cam.ac.uk}}
\vskip -0.5truecm

\keywords{POWHEG, Shower Monte Carlo, NLO, Electroweak}

\abstract{This note documents the use of the package
  \tmtexttt{POWHEG-BOX-RES/vbs-ssww-nloew} for same-sign W-boson scattering processes at the LHC including
  NLO electroweak corrections.
  The generated LH-events can be easily interfaced to shower Monte Carlo programs, in
  such a way that both NLO and shower accuracy are maintained.}
\preprint{\today\\ \tmtexttt{POWHEG-BOX-RES}}

\begin{document}


\section{Introduction}

The \tmtexttt{POWHEG BOX} program is a framework for implementing NLO
calculations in shower Monte Carlo programs according to the
\POWHEG{} method. An explanation of the method and a discussion of 
how the code is organized can be found in Refs.~\cite{Nason:2004rx,Frixione:2007vw,Alioli:2010xd,Jezo:2015aia}.
The code is distributed according to the ``MCNET GUIDELINES for Event Generator
Authors and Users'' and can be found at the web page \\
%
\begin{center}
 \url{http://powhegbox.mib.infn.it}.
\end{center}
%
~\\
%
This program is an implementation of the NLO EW corrections to the
same-sign W-boson scattering process at the LHC  
presented in Ref.~\cite{Chiesa:2019ulk}. Please cite the paper when you use the program.
Though the calculation uses the full LO and NLO matrix elements
for the process ${\rm pp}\to l \nu l' \nu'jj$ with same-sign $l$ and $l'$ 
(at order $\mathcal{O}\left( \alpha^6\right)$ and $\mathcal{O}\left( \alpha^7\right)$, respectively),
the code should only be used for typical VBS selection cuts.
% added-b
This limitation is related to the interface to shower Monte Carlo programs, which perform
the shower evolution of the LH-events preserving the virtuality of the resonances that
appear explicitly in the LH-event files. As the resonance structure of the events written on
disk for $s$-channel processes corresponds to the one in Fig.~1 of Ref.~\cite{Chiesa:2019ulk},
the event selection should suppress the other possible event topologies as in the case of VBS
selection cuts. 
% added-e
In order to run the \tmtexttt{POWHEG BOX} program, we recommend 
to start from the \tmtexttt{POWHEG BOX} user manual, which
contains all the information and settings that are common between
all subprocesses. In this note we focus on  the settings and
parameters specific to the same-sign W-boson scattering implementation.

\section{External libraries}
\label{sec:libraries}

The code relies on the matrix element provider \tmtexttt{RECOLA}~\cite{Actis:2012qn,Actis:2016mpe},
that makes use of the \tmtexttt{COLLIER}~\cite{Denner:2016kdg} library for the evaluation of tensor and scalar
one-loop integrals. The user should download the \tmtexttt{recola2-collier-X.Y.Z} package from
the web page \url{https://recola.hepforge.org/} and  follow the installation instructions therein.  

In addition, we recommend to use {\tt LHAPDF} (\url{https://lhapdf.hepforge.org/})~\cite{Buckley:2014ana} as well as {\tt FastJet} (\url{http://fastjet.fr/})~\cite{Cacciari:2011ma}. These are by default used in our implementation but can be switched off if needed.

We provide an  interface to \tmtexttt{PYTHIA8}~\cite{Sjostrand:2014zea}
(version 8.2XY) for the parton shower evolution and subsequent
hadronization of the events.

In order to build the \tmtexttt{pwhg\_main} executable in \tmtexttt{POWHEG-BOX-RES/vbs-ssww-nloew},
the following  variables  should be set in  the \tmtexttt{Makefile}:
\begin{itemize}
 \item \tmtexttt{RECOLALOCATION=PATH\_TO\_RECOLA2COLLIER/recola2-collier-X.Y.Z/recola2-X.Y.Z} 
 \item \tmtexttt{LHAPDF\_CONFIG=PATH\_TO\_LHAPDF/bin/lhapdf-config}
 \item \tmtexttt{FASTJET\_CONFIG=PATH\_TO\_FASTJET/bin/fastjet-config}
\end{itemize}
For the compilation of the \tmtexttt{main-PYTHIA82-lhef} executable 
the following variables should be defined in the \tmtexttt{Makefile}:
\begin{itemize}
\item \tmtexttt{HEPMCLOCATION=PATH\_TO\_HEPMC}
\item \tmtexttt{PYTHIA8LOCATION=PATH\_TO\_PYTHIA8}
\end{itemize}


\section{Generation of events}
\label{sec:gen_events}



Build the executable\\

\noindent\tmtexttt{\$ cd POWHEG-BOX-RES/vbs-ssww-nloew \\
\$ make pwhg\_main }\\

We provide a sample run folder ({\tt testrun}) to test the event generation at very low statistics:\\

\noindent\tmtexttt{\$
cd testrun\\
\$ ./runpar.sh}\\

The {\tt runpar.sh} script should be edited to export the path to {\tt Recola2}:\\

\noindent\tmtexttt{export LD\_LIBRARY\_PATH=PATH\_TO\_RECOLA2COLLIER/recola2-collier-X.Y.Z/recola2-X.Y.Z}\\

At the end of the run there should be six event files ({\tt pwgevents-000X.lhe}) containing ten
events each. As many upper bound violations occur when running the code at NLO EW accuracy,
a correction factor has to be applied to the weights of the btilde and remnant events
({\tt ub\_btilde\_corr} and {\tt ub\_remn\_corr}, respectively): the {\tt runpar.sh} script
takes care of computing these correction factors after the event generation by running
the {\tt phyton} script {\tt POWHEG-BOX-RES/Scripts/FindReweightFromCounters.py} on the 
{\tt pwgcounters-st4-XXXX.dat} files. The correction factors are then written
in the input files {\tt XXXX-powheg.input} that should be used for the parton shower 
evolution of the events (see the description of the \tmtexttt{main-PYTHIA82-lhef} executable
below). In order to run the parton shower, compile the \tmtexttt{main-PYTHIA82-lhef} executable
and edit the {\tt runpar.sh} script to set the variable {\tt DOPYTHIAPS} to 1.

The second run folder ({\tt run}) contains the files {\tt powheg.input-save} and {\tt runpar.sh}
with all the parameters and settings used to produce the results presented in
Refs.~\cite{Chiesa:2019ulk,Biedermann:2016yds}.\footnote{There is one difference: in the original article,
  the PDF NNPDF3.0QED has been used.   This PDF does not have a {\tt lhaid} identifier.
  Therefore we have put the lhapdf id of NNPDF3.0.} As can be seen from the {\tt runpar.sh}
script, 25 cores are used for the preparation of the grids in stages 1--3 (we actually
generated the grids on a single machine with 5 cores), while 2000 jobs
are used for the event generation in stage 4 (this stage of the calculation was performed
on a computing cluster). Since the 2000 instances of {\tt pwhg\_main} used in stage 4
are independent, the parallelization of the event generation can be easily adapted by
the user.
\\
\\
\\
\\
We provide the executable \tmtexttt{main-PYTHIA82-lhef} which processes the events in the
file \tmtexttt{pwgevents.lhe} and passes them as input to parton shower programs, performing the
required vetoes and setting the required flags in a manner consistent with the aimed (NLO+PS) accuracy
of the input events. The shower Monte Carlo program \PYTHIA{} is used to perform the QCD and QED shower.
The recommended \PYTHIA{} options can be found in the function \tmtexttt{pythia\_init} in the file
\tmtexttt{pythia82F77.cc}: we refer to the online \PYTHIA{} documentation for a description
of the above mentioned options. To shower an event file proceed as follows:
\begin{itemize}
\item build the \tmtexttt{main-PYTHIA82-lhef} executable \\
  \noindent\tmtexttt{\$ cd POWHEG-BOX-RES/vbs-ssww-nloew \\
    \$ make main-PYTHIA82-lhef }\\
\item export the relevant environmental variables \\  
  \noindent\tmtexttt{\$ export PYTHIA8DATA=PATH\_TO\_pythia82XY/share/Pythia8/xmldoc/ \\
  \$ export LD\_LIBRARY\_PATH=\$LD\_LIBRARY\_PATH:PATH\_TO\_HEPMC/lib:PATH\_TO\_PYTHIA/lib/lib }\\
\item run the parton shower on the event file \tmtexttt{pwgevents-XXXX.lhe} using
  the options in the file \tmtexttt{XXXX-powheg.input} \\
  \noindent\tmtexttt{\$ cd testrun \\
    \$ echo \$I > input \\
    \$ echo pwgevents-XXXX.lhe >> input \\
    \$ ../main-PYTHIA82-lhef < input }\\
\end{itemize}
where {\tt \$I} corresponds to index {\tt XXXX} of the event file once the starting {\tt 0} characters have been removed. 
If the event file is generated at NLO EW accuracy, the input file \tmtexttt{XXXX-powheg.input} must
include the input parameters \tmtexttt{ub\_btilde\_corr} and \tmtexttt{ub\_remn\_corr} mentioned above.



\section{Process specific input parameters}
\label{sec:flags}

All the parameters and flags are set in the input file \tmtexttt{powheg.input}.
The mandatory parameters are those needed to select the final-state leptons originating from the vector bosons:
\\
\tmtexttt{
  idvecbos 24 !  24 pp->W+W+jj (-24 pp->W+W+jj)}
\\
\\
\tmtexttt{
  vdecaymode 1 !  1  W(-> e nu\_e)W(-> mu nu\_mu) \\
  \phantom{aaaaaaaaaaaa} ! 2  W(-> e nu\_e)W(-> e  nu\_e ) \\
  \phantom{aaaaaaaaaaaa} ! 3  W(-> mu nu\_mu)W(-> mu nu\_mu)}
\\
\\
In addition to the mandatory parameters, the \POWHEGBOX{} input allows for an easy 
setting of EW and run parameters, by explicitly adding the relevant 
lines to the input card. 
If one of the following entries is not present in the input card  
the reported default value is assumed. The values below correspond to the current default. 
~\\~\\
 \tmtexttt{
   Wmass\phantom{aa}  80.357973609878d0 \phantom{aaaaaaa} ! W pole mass in GeV\\
   Wwidth\phantom{a}  2.0842989982782d0 \phantom{aaaaaaa} ! W pole width in GeV\\
   Zmass\phantom{aa}  91.1534806191827d0 \phantom{aaaaaa} ! Z pole mass in GeV\\
   Zwidth\phantom{a}  2.4942663787728d0 \phantom{aaaaaaa} ! Z pole width in GeV\\
   gmu\phantom{aaaa}  0.11663787d-4 \phantom{aaaaaaaaaaa} ! Fermi constant in GeV\^{}-2\\
   Hmass\phantom{aa}  125d0 \phantom{aaaaaaaaaaaaaaaaaaa} ! Higgs mass in GeV\\
   Tmass \phantom{a} 173.2d0  \phantom{aaaaaaaaaaaaaaaaa} ! top mass in GeV
}
\\
\\
\tmtexttt{runningscale 0\phantom{aaaa}! choice for ren and fac scales in Bbar integration}\\
\tmtexttt{\phantom{aaaaaaaaaaaaaaaaa} ! 0: fixed scale M\_W}\\
\tmtexttt{\phantom{aaaaaaaaaaaaaaaaa} ! 1: running scale sqrt(ptj1*ptj2) }\\ 
The user can implement different functional forms for the running scale editing
the subroutine {\tt set\_fac\_ren\_scales} in the file {\tt Born\_phsp.f}.
\\
\\
\tmtexttt{fakevirt 0 \phantom{aaaaaaa}! 0=off,1=on (default 0)}\\
As the calculation of the virtual matrix elements is highly time-consuming, the user
can replace them with the Born ones multiplied by $\alpha/4\pi$ by setting the flag
{\tt fakevirt} to 1 in the input file. This should only be done in stage 1 (iterations
{\tt xg1} and {\tt xg2} in {\tt runpar.sh}) and stage 3. Using {\tt fakevirt} for stages
2 or 4 spoils the NLO accuracy of the calculation.
\\
\\
\tmtexttt{bornzerodamp 1 \phantom{aaaa}! 0=off,1=on (default 1)}\\
Flag to split the event generation in btilde and remnant generation with improved efficiency (see {\tt POWHEG} manual).
\\
\\
\tmtexttt{rad\_ptsqmin\_em 0.001d0**2 \phantom{aaa} ! minimum pt of the POWHEG radiation \\
 \phantom{rad\_ptsqmin\_em 0.001d0**2Aaaaa}! (default 0.001d0**2) }
\\
\\
\tmtexttt{cmass\_lhe 0d0 \phantom{aaa} ! charm mass written in the LH events (default 0)}\\
Note that quarks are massless in the calculation. The $c$ mass can be used to reshuffle the
event kinematics before the LH event is written on disk.
\\
\\
\tmtexttt{ubexcess\_correct 1 \phantom{aaa} ! store ub-correction factors in \\
 \phantom{ubexcess\_correct 1 aaa} ! pwg-counters-st4-XXXX.dat files (default 0)}\\
Due to the large number on upper-bound violations at NLO, the flag \tmtexttt{ubexcess\_correct} should always
be set to 1 when generating events at NLO EW accuracy. As described above,
the correction factors \tmtexttt{ub\_btilde\_corr} and \tmtexttt{ub\_remn\_corr} are computed by the script
{\tt FindReweightFromCounters.py} and should be added to the input file when running the {\tt PYTHIA}
interface \tmtexttt{main-PYTHIA82-lhef}. 
\\
\\
\subsection{General input flags}

The flags below are common to all the {\tt POWHEG} process-folders. We list them for completeness.
\\
\\
\tmtexttt{
ih1 1 \phantom{aaaaaaaaa}! hadron 1 (1 for protons, -1 for antiprotons)\\
ih2 1 \phantom{aaaaaaaaa}! hadron 2 (1 for protons, -1 for antiprotons)\\
ndns1 131 \phantom{aaaaa}! pdf set for hadron 1 (native mlm numbering) \\
ndns2 131 \phantom{aaaaa}! pdf set for hadron 2 (native mlm numbering) \\
lhans1 260000 \phantom{a}! pdf set for hadron 1 (LHAPDF LHA numbering) \\
lhans2 260000 \phantom{a}! pdf set for hadron 2 (LHAPDF LHA numbering) \\
ebeam1 6500d0 \phantom{a}! energy of beam 1 (GeV)                      \\
ebeam2 6500d0 \phantom{a}! energy of beam 2 (GeV)                      \\
ncall1 10000 \phantom{aa}!st1 number of calls for initializing the integration grid \\
itmx1 2 \phantom{aaaaaaa}!st1 number of iterations for initializing the integration grid \\
ncall2 1000 \phantom{aaa}!st2 number of calls for computing the Bbar integral \\
itmx2 2 \phantom{aaaaaaa}!st2 number of iterations for computing the Bbar integral \\
nubound 1000 \phantom{aa}!st3 number of bbarra calls to setup norm of upper bounding function \\
numevts 10 \phantom{aaaa}!st4 number of events to be generated }\\
\\
\\

\subsection{Flags used by the shower interfaces}

The flags below do not affect the event generation (stages 1--4). They are only
used by the {\tt PYTHIA} interface \tmtexttt{main-PYTHIA82-lhef}.
\\
\\
\tmtexttt{ub\_btilde\_corr 1 \phantom{aaa} ! ub-corr factor for btilde events (default 1)\\
  ub\_remn\_corr 1 \phantom{aaaaa} ! ub-corr factor for remnant events (default 1)}\\
See the description of the flag \tmtexttt{ubexcess\_correct}.
\\
\\
\tmtexttt{use-scalup-ptj 4 \phantom{aaaa}! starting scale for QCD radiation in PS (default 4)\\
  \phantom{aaaaaaaaaaaaaaaaaaaaa}! 0=standard powheg scalup\\
  \phantom{aaaaaaaaaaaaaaaaaaaaa}! 1=hardest quark pt (ptj1)\\
  \phantom{aaaaaaaaaaaaaaaaaaaaa}! 2=softest quark pt (ptj2)\\
  \phantom{aaaaaaaaaaaaaaaaaaaaa}! 3=(ptj1+ptj2)/2\\
  \phantom{aaaaaaaaaaaaaaaaaaaaa}! 4=sqrt(ptj1*ptj2)}\\
Note that the scale is computed from the kinematics of the LH event generated by {\tt POWHEG}.
\\
\\
\tmtexttt{SI\_maxshowerevents -1 !  Number of events to read from the .lhe \\
 \phantom{SI\_maxshowerevents -1a}!  (default: -1, all events)}
\\
\\
\tmtexttt{SI\_pytune  -1  \phantom{aaaaaaaaa}! PYTHIA tune in -1,50 (default -1, default pp tune)\\
  \phantom{SI\_maxshowerevents -1a}! see PYTHIA documentation for the tune numbering}
\\
\\
\tmtexttt{SI\_nohad 0 \phantom{aaaaaaaaaaa}!  if 1 switch off the hadronization (default 0)}
\\
\\
\tmtexttt{SI\_savehistos 1 \phantom{aaaaaa}!  if 1 save the results in a .top file (default 0)}
\\
\\
\tmtexttt{SI\_noQEDq 0 \phantom{aaaaaaaaaa}!  if 1 switch off the QED PS from quarks (default 0)}
\\
\\        
\tmtexttt{SI\_qcdps 1 \phantom{aaaaaaaaaa} ! if 0 switch off QCD PS radiation (defaulf 1)} \\
Note that {\tt PYTHIA} gives an intrinsic $p_{\rm T}$ to the initial-state partons even
for \tmtexttt{SI\_qcdps} equal to 0.
\\
\\
The decays of hadronic resonances which can proceed radiatively have been 
suppressed. In order to allow for such decays, the user should 
open the file \tmtexttt{pythia82F77.cc} and comment the relevant lines in the function {\tt pythia\_init}.

\section*{Acknowledgements}

We would liked to thank Paolo Nason for his help regarding {\sc Powheg}.
MC, AD, and MP acknowledge financial support by the
German Federal Ministry for Education and Research (BMBF) under
contracts no.~05H15WWCA1 and 05H18WWCA1 and the German Research Foundation (DFG) under
reference number DE 623/6-1.
J.-N.~Lang acknowledges support from the Swiss National Science Foundation (SNF)
under contract BSCGI0-157722.
MP is supported by the European Research Council Consolidator Grant NNLOforLHC2.


%\bibliographystyle{JHEP.bst}
\bibliography{manual-BOX-vbs}

\end{document}



